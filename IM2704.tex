\chapter{インピーダンスメータ IM2704}

\section{インピーダンスメータの概要}
\subsection{インピーダンスって何ですか?}

\subsubsection{事実1:オームの法則}
ある抵抗値$R$\footnote{Resistance(抵抗)の頭文字です。}を持つ電気抵抗に流れる電流$I$\footnote{Intensity of Current (電流の強さ)の頭文字でIと略すらしいです。}は、その抵抗にかかる電圧$V$\footnote{Voltage(電圧)の頭文字です。}に比例して大きくなり、抵抗値$R$に反比例して小さくなることが知られています。この関係を式で書くと
$$ I = \frac{V}{R} $$
となります。この法則はオームの法則と呼ばれています。

しかしこの世に「どんな時にも同じ値を出力するもの」なんてそうそうないですよね。
電圧と電流をある時刻$t$における関数として$I(t)$、$V(t)$と表現すると、先のオームの法則は
$$ I(t) = \frac{V(t)}{R} $$
と表すことができ、これも物理現象として広く受け入れられています。

\subsubsection{事実2:交流の電圧・電流}

オームの法則は時刻$t$という一瞬を切り取った時の、電圧と電流と抵抗の関係を表したものでした。

一方、たとえばsin波や矩形波といった周期性をもった電圧・電流については、一瞬の値だけを見ることにあまり意味はありません\footnote{
いまこの瞬間の電圧が0Vだったからといって、最大電圧は100Vなのかもしれないし、10ギガVかもしれません。また、0.1秒後に同じ波形を繰り返す速い波かもしれないし、次の周期が1年後に来る波かもしれません。}。

むしろ、最大値と最小値(振幅)と周期(または周波数)がその物理的性質をよく表していると言えそうです。

このような興味から、時間の枠を少し広げて観察するのがよいと言えるでしょう。
\newline
\newline
\textbf{観察(1) 抵抗の場合}

振幅$A$、角周波数$\omega$の交流電圧$V(t) = A\sin(\omega t)$を抵抗値$R$の抵抗にかけるとします。
この場合、オームの法則から、抵抗に流れる電流$I(t)$は
$$I(t) = \frac{V(t)}{R} = \frac{A}{R}\sin(\omega t)$$
となります。

信号全体としてみても、電流は電圧の$1/R$倍になっていることがわかります。
\newline
\newline
\textbf{観察(2) コンデンサの場合}

振幅$A$、角周波数$\omega$の交流電圧$V(t) = A\sin(\omega t)$を静電容量$C$のコンデンサにかけるとします。

コンデンサには蓄える電荷$Q$と両極にかける電圧$V$、静電容量$C$の間に
$$ Q = CV $$
という関係が成り立つという事実が知られています。

このことと、電流は電荷の時間あたりの変化量であるという事実、すなわち
$$ I = \frac{\partial}{\partial t}Q $$
ということから、コンデンサに流れ込む/から流れ出る電流$I(t)$は、
$$I(t) = \frac{\partial}{\partial t}Q  = \frac{\partial}{\partial t}(CV(t)) = \omega C A\cos(\omega t)$$
とわかります。

今回も信号全体としてみてみます。

振幅大きさを比較するという視点からは、電流は電圧の$\omega C$倍になっていることがわかります。

すなわち、電圧と電流の「振幅」に焦点を当てれば、その逆数の$1/(\omega C)$があたかもオームの法則でいう抵抗のような量となることに気づくと思います。

また、
$$\cos(\omega t) = \sin \left(\omega t + \frac{\pi}{2}\right)$$
であることから、電圧と電流の「位相」に焦点を当てると、電流は電圧に対して$\pi/2$だけ進む(=右に向かう時間軸にプロットすると、同じ時刻のときにすでに先の値を示している=グラフでいえば左側にいる)ことになります。

この、振幅的意味の抵抗が$1/(\omega C)$、位相がずれるのが$+\pi/2$(正の値を進むと解釈)という2つの観察結果を統合して表現できる存在として、\textbf{インピーダンス$Z$}という量が考え出されました。

抵抗にあたる実数値と、ずれる位相の大きさを同時に表現するのに適したものといったら、極座標じゃないでしょうか?

極座標で表せる量は、2次元ベクトルか複素数になるかと思います。
今後「お気持ち」を脳内で考えるときはベクトル、ちゃんと計算するときは複素数と使い分けてやっていこうと思います。

一般的には、インピーダンスは複素数として定義されます。
その大きさが$R$で位相が$\theta$のとき、インピーダンス$Z$は
$$ Z = R(\cos \theta + i \sin \theta)$$
と表すことになります。

ちなみに、オイラーの公式
$$e^{i\theta} = (\cos \theta + i \sin \theta)$$
という数学の公式がありますので、これを用いれば
$$ Z = R e^{i\theta}$$
とも書けます。

\begin{tcolorbox}[title=インピーダンスとアドミッタンス]
    インピーダンスと近い概念に、「アドミッタンス」という概念があります。
    アドミッタンス$Y$はインピーダンス$Z$の逆数$1/Z$として定義されます。

    参考までに、今回みたコンデンサのインピーダンスは
    $$ Z = \frac{1}{\omega C} e^{i\frac{\pi}{2}} = \frac{i}{\omega C}$$
    となりますが、アドミッタンスとして考えても一目瞭然になるよう、分子分母に$i$を掛けて、
    $$ Z = \frac{i}{\omega C} = -\frac{1}{i\omega C} = \frac{1}{Y}$$
    とすることがよくあります。

    最終的に位相やノルムを調べるときは有理化した方がいいですが、並列回路のインピーダンスを考えるときは、逆数をとって足すという過程があるので、こちらの方が便利だったりします。

    わたしが観測した限りでは、インピーダンスの式はアナログ回路担当の人たちはみな当然ように覚えているみたいです。
    わたしは「Qは電流の積分だから…」というところからスタートして計算を辿らないとたどり着けないので、トロくてかなしいです。
\end{tcolorbox}


\textbf{観察(3) コイルの場合}

自己インダクタンス$L$のコイルに正弦波の交流電圧をかけるときのインピーダンスを求めてみます。

コイルは電流$I(t)$の時間変化に比例する、逆向きの起電力$V_{induced}$を生じさせるという性質があります。そのときの比例定数が自己インダクタンス$L$という量で定義されています。

式で表すと
$$V_{induced} = -L\frac{\partial}{\partial t}I(t)$$
という感じになります。

抵抗成分がないと実質無限大の電流が流れてしまうので、抵抗値$R$の抵抗が直列に接続されていると仮定します。
回路に成り立つ式を考えると、1本道の回路ゆえ、電流はどこも同じ量流れるという仮定がおけるので、
$$V(t)-L\frac{\partial}{\partial t}I(t) = R I(t)$$
となります。

この式は、「抵抗+コイルにかける電圧から、誘導起電力の分を差し引いた分が、抵抗Rにかかる電圧になるから、これが電圧降下の$RI(t)$と等しくなるというはずだ」という考えを記述したものです。

この式を移項して、コイルにかかる正味の電圧$V_{coil}(t) = V(t)-RI(t)$に関する式にすると、
$$V(t) - R I(t) = -L\frac{\partial}{\partial t}I(t) \Leftrightarrow V_{coil}(t) = -L\frac{\partial}{\partial t}I(t) $$
と表すことができます。

インピーダンスの定義に則り、$V_{coil}(t)$に$A\sin(\omega t)$をかけたと仮定した場合の、流れる電流について注目します。


あ、これちがうっぽい