\part{序章}

\chapter{アナログ回路の入門書を書くのが\\アナログ回路の専門家のはずがない}


入門書を読むタイミングってどんな時でしょう? お仕事に必要になった、その分野に興味がある、でもその内容がちっともわからない……。
\newline
 アナログ回路強いマン、エッセンスを教えてくれーッ!
\newline
そんな気持ちに応えてくれるのが、世にある数多の入門書たちだと思います。
\newline

ところで、その道のプロは、ちっともわからなかった頃の視点を持ち合わせていて、しかも生々しく保持してくれているでしょうか?

得てして、偉い人は第一線で活躍するのに忙しくて、入門書を書く暇なんてないのです。

書く暇がなかったらどうなるのかって? こんな本が爆誕してしまうわけです。
\newline

本書は、ソフト屋、デジタル屋が頑張って理解した電子回路のエッセンス(?)をまとめた本になる予定です。
アナログ回路の専門家になる手前の皆さんの理解の助けになれば、とてもうれしいです。


\chapter{オームの法則}
「まずはオームの法則から」のように、オームの法則=基本の「き」のように語られることがしばしばあります。したがって、私たちもそこから始めるのはもはや必然なのです。

\section{オームの法則}
「抵抗」という素子について、それに流れる電流$I$とそれに掛かる電圧$V$の間には、
\begin{equation}\label{ohm}
V =IR
\end{equation}
という性質があります。$R$は抵抗素子固有の定数で、抵抗値と呼ばれています。

この式\ref{ohm}には、2通りの解釈方法があると思います。

1つは、「かける電圧が一定ならば抵抗値が大きいほうが電流は小さくなる」という解釈です。例えば、電流を制限しなければならないシチュエーションでは、抵抗値を高めることでそれを達成できそうです。

もう1つは、「流れる電流が一定ならば抵抗値が大きいほうがかかる電圧は大きくなる」という解釈です。シチュエーションに応じて解釈を切り替えながら回路の理解を進めていくのがプロっぽい感じがしますね。

この場ではひとまず、「抵抗素子にある電流$I$が流れているとき、かかる電圧はそれを$R$倍したものになる」と解釈しておきます。

\subsection{分圧}

オームの法則で